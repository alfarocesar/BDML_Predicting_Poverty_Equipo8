\documentclass[12pt,a4paper,onecolumn]{article}

%%%%%%%%%%%%%%%%%%%%%%%%%%%%%%%%%%%
% PAQUETES
%%%%%%%%%%%%%%%%%%%%%%%%%%%%%%%%%%%

\usepackage[margin=1in]{geometry}
\usepackage{authblk}
\usepackage[utf8]{inputenc}  % UTF-8 evita problemas de caracteres
\usepackage[T1]{fontenc}     % Mejor soporte de fuentes en LaTeX
\usepackage[spanish]{babel}  % Manejo correcto de idioma español
\usepackage{amsfonts}
\usepackage{graphicx}
\usepackage{xcolor}
\usepackage{amsmath}
\usepackage{amssymb}
\usepackage[table]{xcolor}
\usepackage{setspace}
\usepackage{booktabs}
\usepackage{dcolumn}
\usepackage{rotating}
\usepackage{threeparttable}
\usepackage[capposition=top]{floatrow}
\usepackage[labelsep=period]{caption}
\usepackage{subcaption}
\usepackage{multicol}
\usepackage[bottom]{footmisc}
\usepackage{enumerate}
\usepackage{units}
\usepackage{placeins}
\usepackage{booktabs,multirow}

% Bibliografía
\usepackage{natbib}
\bibliographystyle{apalike}
\bibpunct{(}{)}{,}{a}{,}{,}

% Formato de párrafos
\renewcommand{\baselinestretch}{1}

% Definir columnas para tablas
\usepackage{array}
\newcolumntype{L}[1]{>{\raggedright\let\newline\\\arraybackslash\hspace{0pt}}m{#1}}
\newcolumntype{C}[1]{>{\centering\let\newline\\\arraybackslash\hspace{0pt}}m{#1}}
\newcolumntype{R}[1]{>{\raggedleft\let\newline\\\arraybackslash\hspace{0pt}}m{#1}}

\usepackage{xfrac}
\usepackage{bbold}

\setcounter{secnumdepth}{6}

\usepackage{titlesec}
\titleformat*{\subsection}{\normalsize \bfseries}

\usepackage[colorlinks=true,linkcolor=black,urlcolor=blue,citecolor=blue]{hyperref}

%%%%%%%%%%%%%%%%%%%%%%%%%%%%%%%%%%%
%     TÍTULO, AUTORES Y FECHA              %
%%%%%%%%%%%%%%%%%%%%%%%%%%%%%%%%%%%

\title{\textbf{Taller 2 - Predicción de Pobreza en Colombia}}

\author{%
\begin{center}

Harold Stiven Acuña\\
José David Cuervo\\
José David Dávila\\
César Augusto Alfaro
\end{center}%
}

\date{\today}

% Configuración simple para espaciado de párrafos
\setlength{\parskip}{0.6em} % Espacio entre párrafos
\setlength{\parindent}{1em} % Sangría moderada

\begin{document}

\maketitle
\thispagestyle{empty}

%%%%%%%%%%%%%%%%%%%%%%%%%%%%%%%%%%%
% ABSTRACT
%%%%%%%%%%%%%%%%%%%%%%%%%%%%%%%%%%%

\begin{abstract}
Este documento presenta el análisis de datos y la implementación de modelos de clasificación para la predicción de la pobreza en Colombia.
\end{abstract}

\medskip

\begin{flushleft}
    {\bf Palabras clave:} pobreza, clasificación, aprendizaje automático \\
    {\bf Clasificación JEL:} J31, C53, J16
\end{flushleft}

% Añadir información del repositorio GitHub
\begin{center}
    \textit{Repositorio GitHub:} \url{https://github.com/alfarocesar/BDML_Predicting_Poverty_Equipo8}
\end{center}

\pagebreak
\doublespacing

%%%%%%%%%%%%%%%%%%%%%%%%%%%%%%%%%%%
%           DOCUMENTO                       %
%%%%%%%%%%%%%%%%%%%%%%%%%%%%%%%%%%%

\section{Introducción}
Breve descripción del problema, antecedentes y motivación del estudio.

\section{Datos}
\subsection{Descripción de la base de datos}
Explicación de la estructura de los datos y variables clave.

\subsection{Procesamiento y limpieza de datos}
Metodologías aplicadas para la adecuación de los datos.

\subsection{Análisis descriptivo}
Exploración de la variabilidad de los datos con tablas y gráficos.

\section{Modelos y Resultados}
\subsection{Metodología}
Descripción de los algoritmos utilizados: Regresión Logística, Random Forest, etc.

\subsection{Entrenamiento y validación de modelos}
Explicación del proceso de entrenamiento y evaluación.

\subsection{Comparación de modelos}
Tabla comparativa con métricas de desempeño.

\subsection{Importancia de variables}
Análisis de las características más relevantes en la predicción.

\section{Conclusión}
Resumen de los hallazgos principales y posibles mejoras futuras.

%%%%%%%%%%%%%%%%%%%%%%%%%%%%%%%%%%%
% TERMINA EL CONTENIDO
%%%%%%%%%%%%%%%%%%%%%%%%%%%%%%%%%%%

\pagebreak
\singlespacing
\nocite{*}
\bibliographystyle{apalike}
\bibliography{references}
\end{document}

%%%%%%%%%%%%%%%%%%%%%%%%%%%%%%%%%%%
% TERMINA EL DOCUMENTO
%%%%%%%%%%%%%%%%%%%%%%%%%%%%%%%%%%%